\documentclass[10pt,a4paper]{article}
\usepackage[utf8]{inputenc}
\usepackage[T1]{fontenc}
\usepackage{amsmath}
\usepackage{amsfonts}
\usepackage{amssymb}
\usepackage{amsthm}
\usepackage{lmodern}
\usepackage{pgf,tikz}
\usepackage{algorithm}
\usepackage{algorithmic}
\usepackage{dsfont}
\usepackage{gensymb}

\theoremstyle{plain}
\newtheorem{theorem}{Theorem}
\newtheorem{lemma}{Lemma}
\newtheorem{corollary}{Corollary}

\title{Homework 2}
\author{Robien \textsc{Clerc} -- Pierre \textsc{Vigier}}

\begin{document}

\maketitle

\section{Randomized analysis}

\textbf{1a} Each vertex is removed with probability $1 - \frac{1}{d}$, thus remains with probability $\frac{1}{d}$

Let $V^*$ the number of remaining vertices we have immediately $\mathbb{E}(V^*) = \frac{n}{d}$

Let $E^*$ the number of remaining edges. An edge remains if and only if the two vertices it connects remain.

Thus we have $\mathbb{E}(E^*) = \frac{n \times d}{2} \times \frac{1}{d^2} = \frac{n}{2d}$

\textbf{2a} Let us consider the graph composed of the remaining vertices and edges from the previous question. It is composed in expectation of $\frac{n}{d}$ vertices and $\frac{n}{2d}$ edges.

We can remove all the remaining edges by removing only one of its two vertices. In expectation there are $\mathbb{E}(V^* - E^*) = \frac{n}{d} - \frac{n}{2d} = \frac{n}{2d}$ vertices remaining and no edges between them : this is an independent set.

There is at least one of those subsets with a value at least the expectation. It proves that G contains an independent set of size $\frac{n}{2d}$

\section{Locality Sensitive Hashing}

\textbf{2a} We suppose that the points are in $\mathbb{R}^2$ and that $\arg(p)$ is the angle in degrees of a point with respect to the $x$ axis.

Let us define sets of angles of width $\alpha$ and offset $\theta$: $S_{\theta} = \{ \theta + \phi \mod 360, \phi \in [0, \alpha[ \}$.

Let us define functions $h_{\theta}$ such that:

$$
h_{\theta}(p) = \left\{\begin{array}{ll}
1 & \text{ if } \arg(p) \in S_{\theta} \\
0 & \text{ otherwise}
\end{array}\right.
$$

Then let $\Theta_1, ..., \Theta_k \overset{\text{i.i.d.}}{\sim} \mathcal{U}([0, 360[)$, we finally define $h$ such that:

$$
h(p) = \underset{i \in \{1, ..., k\}}{\max}{h_{\Theta_i}(p)}
$$

\textbf{2b} First let us find a lower bound on $Pr(h(p) = h(q))$ when $d(p, q) \leq 1\degree$.

\begin{alignat*}{2}
Pr(h(p) \neq h(q)) & = Pr(\{h(p) = 1, h(q) = 0\} \cup \{h(p) = 0, h(q) = 1\}) \\
& = Pr(h(p) = 1, h(q) = 0) + Pr(h(p) = 0, h(q) = 1) \\
& = 2 \times Pr(h(p) = 1, h(q) = 0) \\
& = 2 \times Pr(h(p) = 1) \times Pr(h(q) = 0 | h(p) = 1)
\end{alignat*}

We have that $Pr(h_{\theta_i}(p) = 0) = 1 - \frac{\alpha}{360}$ for all $i$ so:

\begin{alignat*}{2}
Pr(h(p) = 1) & = 1 - Pr(h(p) = 0) \\
& = 1 - Pr(\underset{i \in \{1, ..., k\}}{\max}{h_{\Theta_i}(p)} = 0) \\
& = 1 - Pr(\forall i \in \{1, ..., k\}, h_{\Theta_i}(p) = 0) \\
& = 1 - \prod_{i=1}^k{Pr(h_{\Theta_i}(p) = 0)} \\
& = 1 - \left(1 - \frac{\alpha}{360}\right)^k
\end{alignat*}

Then if $\alpha \geq 1$ then:
\begin{itemize}
\item $Pr(h_{\theta_i}(q) = 0 | h_{\theta_i}(p) = 1) = \frac{\alpha - d(p, q)}{360} \leq \frac{\alpha}{360}$
\item $Pr(h_{\theta_i}(q) = 0 | h_{\theta_i}(p) = 0) = 1 - \frac{d(p, q)}{360} \leq 1$
\end{itemize}

So:

\begin{alignat*}{2}
Pr(h(q) = 0 | h(p) = 1) & = Pr(\forall i \in \{1, ..., k\}, h_{\Theta_i}(q) = 0 | \forall i \in \{1, ..., k\}, h_{\Theta_i}(p)) \\
& = \prod_{i=1}^k{Pr(h_{\Theta_i}(q) = 0 | h_{\Theta_i}(p))} \\
& \leq Pr(h_{\Theta_1}(q) = 0 | h_{\Theta_1}(p) = 1)^{|\{i, h_{\Theta_i}(p) = 1\}|} \\
& \leq Pr(h_{\Theta_1}(q) = 0 | h_{\Theta_1}(p) = 1) \\
& = \frac{\alpha}{360}
\end{alignat*}

Finally, we have that:

$$
Pr(h(p) = h(q)) = 1 - Pr(h(p) \neq h(q)) \geq 1 - \frac{\alpha}{360}\left(1 - \left(1 - \frac{\alpha}{360}\right)^k\right)
$$

\textbf{2c} Now let us find an upper bound on $Pr(h(p) = h(q))$ when $d(p, q) \geq 10\degree$.

\begin{alignat*}{2}
Pr(h(p) = h(q)) & = Pr(\{h(p) = 0, h(q) = 0\} \cup \{h(p) = 1, h(q) = 1\}) \\
& = Pr(h(p) = 0, h(q) = 0) + Pr(h(p) = 1, h(q) = 1)
\end{alignat*}

If $\alpha \leq 10$:

\begin{alignat*}{2}
Pr(h_{\Theta_i}(p) = 0, h_{\Theta_i}(q) = 0) & = Pr(\Theta_i \not \in ]\arg(p)-\alpha, \arg(p)] \cup ]\arg(q)-\alpha, \arg(q)]) \\
& = 1 - Pr(\Theta_i \in ]\arg(p)-\alpha, \arg(p)] \cup ]\arg(q)-\alpha, \arg(q)]) \\
& = 1 - (Pr(\Theta_i \in ]\arg(p)-\alpha, \arg(p)]) + Pr(\Theta_i \in ]\arg(q)-\alpha, \arg(q)])) \\
& = 1 - \frac{2\alpha}{360}
\end{alignat*}

So:

\begin{alignat*}{2}
Pr(h(p) = 0, h(q) = 0) & = Pr(\forall i \in \{1, ..., k\}, h_{\Theta_i}(p) = 0, h_{\Theta_i}(q) = 0) \\
& = \prod_{i=1}^k{Pr(h_{\Theta_i}(p) = 0, h_{\Theta_i}(q) = 0)} \\
& = \left(1 - \frac{2\alpha}{360}\right)^k
\end{alignat*}

Let us remark that $Pr(h_{\Theta_i}(p) = 1, h_{\Theta_i}(q) = 1)) = 0$ when $\alpha \leq 10$. Thus:

\begin{alignat*}{2}
Pr(h(p) = 1, h(q) = 1) & = Pr(\exists i \neq j \in \{1, ..., k\}, h_{\Theta_i}(p) = 1, h_{\Theta_j}(q) = 1) \\
& = \sum_{A \subsetneq \{1, ..., k\}, A \neq \emptyset}{Pr(\forall i \in A, h_{\Theta_i}(p) = 1, \exists j \not \in A, h_{\Theta_j}(q) = 1)} \\
& = \sum_{A \subsetneq \{1, ..., k\}, A \neq \emptyset}{Pr(\forall i \in A, h_{\Theta_i}(p) = 1)Pr(\exists j \not \in A, h_{\Theta_j}(q) = 1)} \\
& = \sum_{A \subsetneq \{1, ..., k\}, A \neq \emptyset}{\left(\frac{\alpha}{360}\right)^{|A|}(1 - Pr(\forall j \not \in A, h_{\Theta_j}(q) = 0))} \\
& = \sum_{A \subsetneq \{1, ..., k\}, A \neq \emptyset}{\left(\frac{\alpha}{360}\right)^{|A|}\left(1 - \left(1 - \frac{\alpha}{360}\right)^{k-|A|}\right)} \\
& = \sum_{l = 1}^{k-1}{{k \choose l}\left(\frac{\alpha}{360}\right)^{l}\left(1 - \left(1 - \frac{\alpha}{360}\right)^{k-l}\right)} \\
& \leq \sum_{l = 0}^{k}{{k \choose l}\left(\frac{\alpha}{360}\right)^{l}\left(1 - \left(1 - \frac{\alpha}{360}\right)^{k-l}\right)} \\
& = \sum_{l = 0}^{k}{{k \choose l}\left(\frac{\alpha}{360}\right)^{l}} - \sum_{l = 0}^{k}{{k \choose l}\left(\frac{\alpha}{360}\right)^{l}\left(1 - \frac{\alpha}{360}\right)^{k-l}} \\
& = \left(1 + \frac{\alpha}{360}\right)^{k} - 1
\end{alignat*}

Hence:

$$Pr(h(p) = h(q)) \leq \left(1 - \frac{2\alpha}{360}\right)^k + \left(1 + \frac{\alpha}{360}\right)^{k} - 1
$$

\textbf{2d} To sum up, we have shown that:

\begin{alignat*}{3}
Pr(h(p) = h(q)) & \geq 1 - \frac{\alpha}{360}\left(1 - \left(1 - \frac{\alpha}{360}\right)^k\right) & \text{ when } d(p, q) \leq 1\degree \\
Pr(h(p) = h(q)) & \leq \left(1 - \frac{2\alpha}{360}\right)^k + \left(1 + \frac{\alpha}{360}\right)^{k} -  1 & \text{ when } d(p, q) \geq 10\degree \\
\end{alignat*}

To conclude we have to find $\alpha \in [0, 10]$ and $k \geq 1$ such that $Pr(h(p) = h(q)) \geq 0.99$ and $Pr(h(p) = h(q)) \leq 0.95$.

If we take $\alpha = 10\degree$ and $k = 10$, we have that:

\begin{alignat*}{1}
1 - \frac{\alpha}{360}\left(1 - \left(1 - \frac{\alpha}{360}\right)^k\right) \approx 0.993 \\
\left(1 - \frac{2\alpha}{360}\right)^k + \left(1 + \frac{\alpha}{360}\right)^{k} -  1 \approx 0.880 \\
\end{alignat*}

\end{document}