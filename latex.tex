\documentclass[10pt,a4paper]{article}
\usepackage[utf8]{inputenc}
\usepackage[T1]{fontenc}
\usepackage{amsmath}
\usepackage{amsfonts}
\usepackage{amssymb}
\usepackage{amsthm}
\usepackage{lmodern}
\usepackage{pgf,tikz}
\usepackage{algorithm}
\usepackage{algorithmic}
\usepackage{dsfont}

\theoremstyle{plain}
\newtheorem{theorem}{Theorem}
\newtheorem{lemma}{Lemma}
\newtheorem{corollary}{Corollary}

\title{Homework 2}
\author{Robien \textsc{Clerc} -- Pierre \textsc{Vigier}}

\begin{document}

\maketitle

\section{Randomized analysis}

\textbf{1a} Each vertex is removed with probability $1 - \frac{1}{d}$, thus remains with probability $\frac{1}{d}$

Let $V^*$ the number of remaining vertices we have immediately $\mathbb{E}(V^*) = \frac{n}{d}$

Let $E^*$ the number of remaining edges. An edge remains if and only if the two vertices it connects remain.

Thus we have $\mathbb{E}(E^*) = \frac{n \times d}{2} \times \frac{1}{d^2} = \frac{n}{2d}$

\textbf{2a} Let us consider the graph composed of the remaining vertices and edges from the previous question. It is composed in expectation of $\frac{n}{d}$ vertices and $\frac{n}{2d}$ edges.

We can remove all the remaining edges by removing only one of its two vertices. In expectation there are $\mathbb{E}(V^* - E^*) = \frac{n}{d} - \frac{n}{2d} = \frac{n}{2d}$ vertices remaining and no edges between them : this is an independent set.

There is at least one of those subsets with a value at least the expectation. It proves that G contains an independent set of size $\frac{n}{2d}$




\end{document}