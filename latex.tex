\documentclass[10pt,a4paper]{article}
\usepackage[utf8]{inputenc}
\usepackage[T1]{fontenc}
\usepackage{amsmath}
\usepackage{amsfonts}
\usepackage{amssymb}
\usepackage{amsthm}
\usepackage{lmodern}
\usepackage{pgf,tikz}
\usepackage{algorithm}
\usepackage{algorithmic}
\usepackage{dsfont}
\usepackage{gensymb}

\theoremstyle{plain}
\newtheorem{theorem}{Theorem}
\newtheorem{lemma}{Lemma}
\newtheorem{corollary}{Corollary}

\title{Homework 2}
\author{Robien \textsc{Clerc} -- Pierre \textsc{Vigier}}

\begin{document}

\maketitle

\section{Randomized analysis}

\textbf{1a} Each vertex is removed with probability $1 - \frac{1}{d}$, thus remains with probability $\frac{1}{d}$.

Let us define for each $v \in V$, $X_v = \mathds{1}_{v\text{ remains}}$.

Let $N_V = \sum_{v \in V}{X_v}$ be the number of remaining vertices. Then, we immediately have:

$$
\mathbb{E}(N_V) = \sum_{v \in V}{\mathbb{E}(X_v)} = n Pr(v \text{ remains}) = \frac{n}{d}
$$

Let us define for each $e \in E$, $Y_e = \mathds{1}_{e\text{ remains}}$. But an edge remains if and only if the two vertices it connects remain so for each $e = (u, v)$, $Y_e = X_uX_v$.

Let $N_E = \sum_{e \in E}{Y_e}$ be the number of remaining edges. 

We have by independence of the $(X_v)_{v \in V}$ that $\mathbb{E}(Y_e) = \mathbb{E}(X_u) \mathbb{E}(X_v) = \frac{1}{d^2}$.

Thus we have $\mathbb{E}(N_E) = \frac{n \times d}{2} \times \mathbb{E}(Y_e) =\frac{n \times d}{2} \times \frac{1}{d^2} = \frac{n}{2d}$

\textbf{2a} Let us consider the graph composed of the remaining vertices and edges from the previous question. It is composed in expectation of $\frac{n}{d}$ vertices and $\frac{n}{2d}$ edges.

We can remove all the remaining edges by removing only one of its two vertices. In expectation there are $\mathbb{E}(N_V - N_E) = \frac{n}{d} - \frac{n}{2d} = \frac{n}{2d}$ vertices remaining and no edges between them : this is an independent set.

There is at least one of those subsets with a value at least the expectation. It proves that G contains an independent set of size $\frac{n}{2d}$

\section{Locality Sensitive Hashing}

\textbf{2a} We suppose that the points are in $\mathbb{R}^2$ and that $\arg(p)$ is the angle in degrees of a point with respect to the $x$ axis.

Let us define sets of angles of width $\alpha$ and offset $\theta$: $S_{\theta} = \{ \theta + \phi \mod 360, \phi \in [0, \alpha[ \}$.

Let us define functions $h_{\theta}$ such that:

$$
h_{\theta}(p) = \left\{\begin{array}{ll}
1 & \text{ if } \arg(p) \in S_{\theta} \\
0 & \text{ otherwise}
\end{array}\right.
$$

Let $\mathcal{H}$ be the family generated when $\Theta \sim \mathcal{U}([0, 360[)$

\textbf{2b} First let us find a lower bound on $Pr(h(p) = h(q))$ when $d(p, q) \leq 1\degree$.

\begin{alignat*}{2}
Pr(h(p) \neq h(q)) & = Pr(\{h(p) = 1, h(q) = 0\} \cup \{h(p) = 0, h(q) = 1\}) \\
& = Pr(h(p) = 1, h(q) = 0) + Pr(h(p) = 0, h(q) = 1) \\
& = 2 \times Pr(h(p) = 1, h(q) = 0) \\
\end{alignat*}

We have that $Pr(h(p) = 1) = \frac{\alpha}{360}$. Moreover $Pr(h(q) = 0 | h(p) = 1) = \frac{d(p, q)}{\alpha} \leq \frac{1}{\alpha}$. So:

$$
Pr(h(p) = 1, h(q) = 0) = Pr(h(p) = 1)Pr(h(q) = 0 | h(p) = 1)\leq \frac{1}{360}
$$

Finally, we have that:

$$
Pr(h(p) = h(q)) = 1 - Pr(h(p) \neq h(q)) \geq 1 - \frac{2}{360} \geq 0.99
$$

\textbf{2c} Now let us find an upper bound on $Pr(h(p) = h(q))$ when $d(p, q) \geq 10\degree$.

\begin{alignat*}{2}
Pr(h(p) = h(q)) & = Pr(\{h(p) = 0, h(q) = 0\} \cup \{h(p) = 1, h(q) = 1\}) \\
& = Pr(h(p) = 0, h(q) = 0) + Pr(h(p) = 1, h(q) = 1)
\end{alignat*}

If $\alpha \leq 10$:

\begin{alignat*}{2}
Pr(h(p) = 0, h(q) = 0) & = Pr(\Theta \not \in ]\arg(p)-\alpha, \arg(p)] \cup ]\arg(q)-\alpha, \arg(q)]) \\
& = 1 - Pr(\Theta \in ]\arg(p)-\alpha, \arg(p)] \cup ]\arg(q)-\alpha, \arg(q)]) \\
& = 1 - (Pr(\Theta \in ]\arg(p)-\alpha, \arg(p)]) + Pr(\Theta_i \in ]\arg(q)-\alpha, \arg(q)])) \\
& = 1 - \frac{2\alpha}{360}
\end{alignat*}

Then, let us remark that $Pr(h(p) = 1, h(q) = 1)) = 0$ when $\alpha \leq 10$.

Hence:

$$
Pr(h(p) = h(q)) \leq 1 - \frac{2\alpha}{360}
$$

\textbf{2d} To conclude we have to find $\alpha \leq 10$ such that $1 - \frac{2\alpha}{360} \leq 0.95$.

If we take $\alpha = 10\degree$ then $1 - \frac{2\alpha}{360} \approx 0.944$.

\end{document}